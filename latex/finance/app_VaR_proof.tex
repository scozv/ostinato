\section{Proof of VaR}
\subsection{满足正态分布的收益率的VaR}
假定期初资产投资额为$S$,一定时期$t$內的收益率$r$
服从正态分布$N(\mu, \sigma^2)$,则置信水平$\alpha$下的风险价值为:
\marginpar{课堂上的含$\Phi^{-1}(\alpha)$的公式是错误的,随后证明}
\[\text{VaR} = -S[\mu + \sigma \cdot \Phi^{-1}(1-\alpha)] \]
\subsection{理解和证明}
通常的,在时期$t$内的置信度是一个
较大($>50\%$)的数,比如$99\%$,我们想要表达的
语言是:

\emph{在时期$t$内,有$1-\alpha$的可能性,使我们的损失
超过$\text{VaR}$}

\marginpar{文献\citef{J.C.Hall:2001}给出了损失的
两种字面表达方式:
收益曲线中的负收益,损失曲线中的正损失,一个意思,
我们更希望用数学语言表示}

令$S$为资产期初值,期间收益率$r$,则期末资产的价值$W$:
\[ W=S(1+r)\]
于是期间内资产的变化值$\Delta p$(正值为收益,负值为损失):
\[ \Delta p = W-S=S(1+r)-S=S\cdot r\]
特别注意,上述等式中,\emph{$r$是随机变量},满足一定的概率分布,
$\Delta p=S\cdot r$是随机变量的函数,也是一个随机变量。
如果$r$服从$N(\mu, \sigma^2)$的正态分布,
则$\Delta p = S\cdot r$服从$N(S\mu, S^2 \sigma^2)$的正态分布,

我们约定,价值$\text{VaR}$是一个非负数($\text{VaR}\geq 0$),
当我们说“损失超过$\text{VaR}$”的时候,对应的数学语言是:
\[ |\Delta p| \geq \text{VaR}, (\Delta p \leq 0) \]

基于这些记号(Notation),上面的VaR描述的更精确定义为:
\begin{equation}
  P(|\Delta p| \geq \text{VaR}) = 1-\alpha, (\Delta p \leq 0)
\end{equation}

下面就是推导这个式子,
假定$r\sim N(\mu, \sigma^2), \Delta p = S\cdot r$:

\begin{align*}
\intertext{已知:}
&P(|\Delta p| \geq \text{VaR}) = 1-\alpha \\
\intertext{其中$\Delta p \leq 0$ 并且$\text{VaR} \geq 0 $}
\Rightarrow &P(-\Delta p \geq \text{VaR}) = 1-\alpha \\
\intertext{变化一下随机变量$\Delta p$ 的符号}
\Rightarrow &P(\Delta p \leq -\text{VaR}) = 1-\alpha \\
\intertext{我们已经说过$\Delta p\triangleq Sr$:}
\Rightarrow &P(r \leq -\frac{\text{VaR}}{S}) = 1-\alpha \\
\intertext{已知$r\sim N(\mu, \sigma^2)$,将其标准化:}
\Rightarrow &P(\frac{r-\mu}{\sigma} \leq \frac{-\frac{\text{VaR}}{S}-\mu}{\sigma}) = 1-\alpha \\
\intertext{根据标准正态分布的定义和性质:}
\Rightarrow &\Phi(\frac{-\frac{\text{VaR}}{S}-\mu}{\sigma}) = 1-\alpha \\
\Rightarrow &\Phi^{-1}(1-\alpha) = \frac{-\frac{\text{VaR}}{S}-\mu}{\sigma} \\
\intertext{整理方程,求解$\text{VaR}$:}
\Rightarrow &\text{VaR} = -S[\mu + \sigma \cdot \Phi^{-1}(1-\alpha)] \\
\intertext{根据标准正态分布累计密度函数$\Phi(x)$的性质:}
\Rightarrow &\text{VaR} = -S[\mu - \sigma \cdot \Phi^{-1}(\alpha)]
\end{align*}

\subsection{额外的证明}
\begin{proof}[$\Phi^{-1}(1-\alpha)=-\Phi^{-1}(\alpha)$]
\begin{align*}
\intertext{由标准正态分布概率密度函数的图像对称性:}
&\alpha \triangleq \Phi(x_0) \\
\Rightarrow & \alpha = 1-\Phi(-x_0) \\
\Rightarrow &
     \left\{  \begin{array}{l}
     \Phi(x_0) = \alpha \\
     \Phi(-x_0) = 1-\alpha
    \end{array}  \right. \\
\Rightarrow &
     \left\{  \begin{array}{l}
     \Phi^{-1}(\alpha) = x_0 \\
     \Phi^{-1}(1-\alpha) = -x_0
    \end{array}  \right. \\
\Rightarrow & 
  \Phi^{-1}(1-\alpha)=-\Phi^{-1}(\alpha) \qedhere
\end{align*}
\end{proof}

\subsection{两个说明}
\begin{enumerate}
  \item 课堂上给出的公式为什么是:$\text{VaR} = -S[\mu + \sigma \cdot \Phi^{-1}(\alpha)]$?
	
	这个公式是错误的,如果使用文献\citef{J.C.Hall:2001}第9章的例9.1验证(验证过程随后给出),
	会发现该公式的计算结果和文献的结果不一致,但是本节推导出来
	的公式可以求得正确的VaR值;
  \item 为什么课堂上的例子$\text{VaR}=-120 \times 1000 \times (-1.65) \times 2\%$ 可以套用错误的公式?
	
	因为这个例子中恰好$\mu =0$ ,图像相对于纵轴是对称的。需要注意的是
	$-1.65$不是计算的$\Phi^{-1}(\alpha)$,然后添加一个负号,让VaR取正。
	$-1.65$计算的是$\Phi^{-1}(1-\alpha)$,课堂上有同学提出的疑问是有道理的。
\end{enumerate}

\subsection{公式验证}
文献\citef{J.C.Hall:2001}第9章的例9.1如是说:

\begin{quote}
Suppose that the gain from a portfolio during six months is normally distributed with
a mean of $\$2$ million and a standard deviation of $\$10$ million. From the properties of
the normal distribution, the one-percentile point of this distribution is $2 − 2.33 × 10$
or $–\$21.3$ million. The VaR for the portfolio with a time horizon of six months and
confidence level of $99\%$ is therefore $\$21.3$ million.
\end{quote}

课件上给出的公式是$\text{VaR} = -S[\mu + \sigma \cdot \Phi^{-1}(\alpha)]$,根据题意:
\begin{align*}
S &\triangleq 1 \\
\mu &= 2 \text{ million} \\
\sigma & = 10 \text{ million} \\
\alpha & = 99\% \\
\intertext{代入课件的公式:}
\text{VaR} &= -1\cdot [2 + 10 \cdot \Phi^{-1}(99\%)] \\
\intertext{查表(\citef{GB4086.1:83})得$\Phi^{-1}(99\%) \approx 2.33$:}
\Rightarrow \text{VaR} &= -1\cdot [2 + 10 \times 2.33] = -25.3 
\end{align*}

这个结果不等于例9.1的计算结果。

\subsection{$\Delta p \triangleq -Sr$对公式的影响}
接下来会证明,即使定义$\Delta p \triangleq -Sr$,课件中给的公式依然有问题。

回顾文字定义:\emph{在时期$t$内,有$1-\alpha$的可能性,使我们的损失
超过$\text{VaR}$}。

如果将风险价值定义为期初值减去期末值,也就是
$\text{VaR} \triangleq S - S(1+r) = -Sr$,我们默认地\emph{将VaR看成了一个负值},
当我们有正的收益的时候,$\Delta p = -Sr$应该是一个负值,所以,当我们再来说
损失的时候,我们又默认地假定了$\Delta p \geq 0$。

这两个默认假定对我们的证明非常重要,因为,对应的数学定义变成了:
\begin{align*}
  &\Delta p \triangleq -Sr \\
  \intertext{其中$S > 0, r \sim N(\mu, \sigma^2)$}
  &P(|\Delta p| \geq |\text{VaR}|) = 1-\alpha, (\Delta p \geq 0, \text{VaR} \leq 0)\\
  \intertext{展开绝对值,注意符号:}
  &P(\Delta p \geq -\text{VaR}) = 1-\alpha\\
  \Rightarrow &P(-Sr\geq -\text{VaR}) = 1-\alpha \\
  \Rightarrow &P(Sr\leq \text{VaR}) = 1-\alpha \\
  \Rightarrow &P(r\leq \frac{\text{VaR}}{S}) = 1-\alpha, (S > 0) \\
  \Rightarrow &\Phi(\frac{\frac{\text{VaR}}{S}-\mu}{\sigma}) = 1-\alpha \\
  \Rightarrow &\Phi^{-1}(1-\alpha) = \frac{\frac{\text{VaR}}{S}-\mu}{\sigma} \\
  \Rightarrow &\text{VaR} = S\left[\mu+\sigma\cdot\Phi^{-1}(1-\alpha)\right] \\
  \intertext{根据$\Phi^{-1}(1-\alpha)=-\Phi^{-1}(\alpha)$:}
  \Rightarrow &\text{VaR} = S\left[\mu-\sigma\cdot\Phi^{-1}(\alpha)\right] \\
\end{align*}

代入例9.1的数据,得出VaR是一个负值,这符合我们对VaR的假设。
同时这个公式也指出了课件的公式$\text{VaR} = -S[\mu + \sigma \cdot \Phi^{-1}(\alpha)]$是错误的。

总结一下,不管如何假设$\Delta p$的正负号定义,本节推导出来了两个公式:
\begin{align*}
\text{VaR} &= -S[\mu - \sigma \cdot \Phi^{-1}(\alpha)] \\
\text{VaR} &= S\left[\mu-\sigma\cdot\Phi^{-1}(\alpha)\right]
\end{align*}

课件给出的公式是:
\begin{equation*}\text{VaR} = -S[\mu + \sigma \cdot \Phi^{-1}(\alpha)]\end{equation*}

它们之间的区别不在于$S$和$-S$的区别,而在于$\sigma \cdot \Phi^{-1}(\alpha)$前面的符号。
课堂上给出的期权VaR的例子,VaR计算结果的绝对值之所以能够满足所有的等式,原因是题目给出的
平均值$\mu$是零。