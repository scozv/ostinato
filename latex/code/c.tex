\section{The C Programming Language}
书籍对应\citec{B.W.Kernighan:1988}和对应的汉译本。

% C 中文2新版, 所有的页码以Reader为准
\begin{lstlisting}
// C中的所有字符都对应一个int,“但由于潜在原因,我们选用int”(页10)
int c;

c = getchar();
while (c!=EOF) {
  putchar(c);
  c = getchar();
}
\end{lstlisting}

在C中,赋值操作是一个表达式——具有一个值,所以:
\begin{lstlisting}
int c;

while ((c=getchar()) != EOF) {
  putchar(c);
}
\end{lstlisting}
P12
long至少占32位,某些机器上int和long的长度相同,
但有些机器的int只有16位[-32768~32767]。
double可以处理更大的数,可以使用\cd{"\%.0f"}只打印整数部分。

P19 C1.8参数
在C中,所有函数参数都是ByValue传递的——也就是说,
The value passed into function as parameter is stored in temporary variable, 
not in the previous variable before function calling.
