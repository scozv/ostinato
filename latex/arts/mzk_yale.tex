\section{Introduction to Classical Music}
\marginpar{前七讲,到和弦部分根据Yale大学旧版视频整理,
2015年初开始聆听Coursera的新版公开课。这部分的subsection根据
新版的Course Syllabus整理}

参考书目:Wright, The Essential Listening to Music

\subsection{Music in Our Lives}
what is music, art is life by Yo-yo ma. life is imagination.
music is the rational organization
of sounds and silences passing
through time.

"Music is what feelings sound like" by Victor Hugo.

bond to music, I'm part of this group.
\marginpar{J.S Bach, Cantata BWV 140}

\subsubsection{Pop vs. Classical}
\begin{tabular}{l|p{4cm}|p{4cm}}\hline
 & Pop & Classical \\ \hline
Instruments & often uses electronic instruments & acoustic instruments  \\ \hline
Lyric & includes lyrics for emotion & just music except opera  \\ \hline
Beat & often has a strong beat & more subdued (suppressed)  \\ \hline
Duration & short and repetitious & 
long  \\ \hline
Natation & without musical notation in performing & score  \\ \hline
\end{tabular}
\marginpar{Beats: "We Got It Goin' On" by Bon Jovi, Tchckv PC No.1 in B Flat minor, 1875}

\marginpar{Stravinsky, The Rite of Spring, 1913}
\marginpar{Mozart, Eine Kleine Nachtmusik, 1787}
\marginpar{J.S Bach, BWV 578, 1703-07}

低音震动频率低,高音震动频率高。

\subsubsection{Nurture and Nature}
Nurture就是指生在什么环境。

Here are a couple of rules for the syntax of Western music:
\begin{itemize}
  \item Leading Tone: pulls toward the home pitch or tonic
  \item Large Leap: that is reversed by the following pitch
  \item Harmony: that must fit with the Melody
  \item Chord Progression: the harmonies usually return hame to tonic
\end{itemize}

The natural consonance of the overtone series:
\begin{itemize}
  \item the overtone series stretches upward across several octaves
  \item it includes many consonances as well as potential dissonances
\end{itemize}
\subsection{Rhythm that We have All Got It}
\subsubsection{Beats}
\marginpar{Strauss, Also Sprach Zarathustra, 1896}
\marginpar{"Classical music is music without africa" by Brain Eno}
\marginpar{Elgar, "Pomp and Circumstance"}
Grouping Beats:
\begin{itemize}
  \item undifferentialed beats convey a consistent pulse
  \item often, our brains subconsciously group pulse into units of 2,3, or 4
  \item these grouping lead to what we call \emph{meter} in music
\end{itemize}
\subsubsection{Rhythm}
the coordinates of music: pitch and duration.

\marginpar{全音符,半分音符等}

五线谱中:
\begin{itemize}
  \item Higher and lower positions correspond to higher and lower sounds
  \item Lighter and darker, or 'white' and 'black' notes, correspond to 
  slower and faster rhythms
\end{itemize}
\marginpar{Bolero, as Example}

\subsubsection{Tempo}
Tempo in music is simply the speed of the beat.

\begin{tabular}{l|r}\hline
Tempo Marking & Speed (BPM) \\ \hline
Lento & 40-45 \\ \hline
Largo & 45-50 \\ \hline
Adagio & 55-65 \\ \hline
Andante & 73-77 \\ \hline
Moderato & 86-97 \\ \hline
Allegretto & 98-109 \\ \hline
Allegro & 109-132 \\ \hline
Vivace & 132-140 \\ \hline
Presto & 168-177 \\ \hline
Prestissimo & > 178 \\ \hline
\end{tabular}

\subsection{What is Melody}
中国传统音乐中一个音阶有五个音(pentatonic scale),印度有六个。西方古典 end up
with seven pitches in our scale.
古希腊用数学来解释万物,包括音乐:ABCDEFG。

\subsubsection{Major and Minor}
The major and minor scales, involve two different sequences of
intervals. In each major and minor scale, there are five adjacent
intervals.

Scale ladders:
\begin{lstlisting}
        1      2      3   4      5      6      7   8
Major: -+------+------+---+------+------+------+---+->
Minor: -+------+---+------+------+---+------+------+->
        1      2   3      4      5   6      7      8
\end{lstlisting}

\subsubsection{Chromatic Scale}
The chromatic scale uses all 12 possible pitches, 7个白键
和其中的五个黑键。

\subsubsection{Melodic Structure}
The \emph{tonic} is always the first note of the scale.
Either way, you're using a scale, major or minor, beginning on C.

\subsubsection{Modulation}
Composers can change keys. When they change keys, 
they affect what is called a modulation.
\marginpar{Copland's Appalachian Spring}

\subsubsection{Phrase Structure}
Antecedent consequence phrase structure is endemic 
to melody in all western music.

Lec 01 元素

	
lobe temporal → 听觉处理中心   				贝五交响
大调 小调,跳跃					                九		
home pitch
走向 → 没有鼓动节奏 going down				德彪西,月光

dimenion of music: pitch 音高, time (duration) 时长


音高
Ex: 查拉图斯特拉如是说,理查斯特劳斯

时长,节拍
Ex: John Kander 芝加哥 二拍,三拍

转调 modulation


Lec 02 形式


movement 独立篇章
genre /ˈZHänrə/: 交响,协奏,tone poem(历史史诗,Ex 如是说)等

贝五交响 mvt 4, 转向大调

柴一钢协 mvt 1
motive at beginning, minor
钢琴弹奏和弦 motive
弦乐 theme
pizzicato 弦,钢琴 theme

泛音,圆号
巴松管,彼得和狼中的祖父
小提琴 vibrato 揉弦 碎弓

图片展览会 polish oxcar,音量从小至大,再变小,多普勒效应
低音波长长

Ex: 理查 斯特劳斯,死亡和升华
和谐和弦,不和谐(频率相似),解构


Lec 03

	 五线谱参考lec03.mscz 时长


Aim
music notation:优点,precisely表达细节,presever保存原作;劣势,无法记录复杂旋律,如Jazz即兴
half notes, quarter notes...

Ex: 穆斯林圣歌,Chuck Mangione

notes 表示时间

各小节的时长一样,从左到右分别是:
whole note 全音符,half note 二分音符,四分音符 quater note,八分音符 eighth note
最后一小节用了一个dot,表示前面音符的50%,
图片没有给出rests(休止符)

pulse beat rythem in music 
流行音乐通常将beat rythem放在首位

四分音符通常carries the beat

长短不同的整合 构成 rythm

Cole Porter 斗牛士 bass 低音区分二拍(SW,强弱,指挥down up, down up)或者三拍(SWW,强弱弱,指挥 d over up, down over up)

syncopation,切分音,切分节奏,提前介入,并在短时间内打破拍子的平衡

pick up 强音,down beat 强音,所有音乐都有down beat,但不总是从down beat开始

如何将重音传递出来:
延长 duration
down音时,向下拉弦乐
音域 range
和弦 chords变化,和弦的变化常常在重音

Bolero的乐谱?


Lec 04 切分音、三连音、织体


开场游戏音乐,甲壳虫Lucy In The Sky With Diamonds

fundamental pitch, overtones

节拍 pulse 律动
节奏 rythm

tempo → beat speed: ritardando

切分音 Synkope


Ex: Scott Japin 切分音大师《艺人》

三连音
Ex: 英国国歌,god save the king


左手打二拍,右手同样的时间打三拍

musical taxture
monophonic texture, homophonic …, polyphonic …
单音织体,主音织体,复音织体(对位法)

其中复音织体(对位法):imitative模仿式,音乐转位,倒影模仿;fell 自由对位(Ex: Louis Armstrong) 

Ex: 《奇异恩典》Amazng Grace

自由对位:Ex Johony Dodd, 哭泣的威利

听力练习,图画展览会,基辅的城门

小提琴碎弓,小号三连音符

Mozart 安魂曲 末日经 Apocalypse Contutatis
Lacrimosa, out of the Dies irae, out of the Requiem Mass of W.A. Mozart, Vienna, 1791


Lec 05 melody, pitch of texture

melody: easier to heard from high range pitch 声波传播方式,频率更高,更短消失,更清晰

opus 7 sonata of beethoven, 1797 year

western naotation, octave 八度 两倍频率 duplication ovtave

六阶 sitar 印度 ravi shava who is the father of nona johons

布鲁斯 六阶八度

五阶八度 中国二胡

leimma 小半音,音阶,大调模式,小调模式 scale,大调小调来自传统国歌

七音音阶

tonic tone ← leading tone

小调古典音乐不多,贝三交响第二乐章

小三度 《逃狱三王》

半音音阶 mozart 安魂曲

级进 欢乐颂,跳进

织体,单音,复音,复合

对位法


Lec 06 Mozart和瓦格纳

antecedant, conseque前置句,后置句
《玛卡丽娜》的主旋律 
syntax of …

普西尼,《艺术家的生涯》(《吉屋出租》据此改编)
咏叹,《亲爱的爸爸》八度octave的跳进,回到主音tonic

终止式:
正格终止,
假终止deceptive cadence,配上一个和音,flyover

旋律模进 melodic sequence is simply the repetition of a musical motive at a successively higher or lower degree of scale

<Tridan and Isolde> 瓦格纳 1865
knight and pricess
overtune,思考什么乐器(string and woodwind),什么模式

三小时之后的love-death,改变作曲方式:
曲调柔和,音域降低,不协和音转协和,旋律模进,变格终止(阿门终止),假终止

Voi che sapeta by Mozar 1786,选自费加罗,上行模进。由Lauren Libaw演唱,她曾和Isaac Stern的儿子David Stern合作演出

前置句 后置

结构和对称symentry在M的作品中有重要的地位

歌词中,donne vedetre slio lihi nel cor 来到了一个假终止


Lec 07 和弦(Harmony)和如何创建主题

影响西方音乐的有:
手写五线谱
和音 chrod——西方音乐特有

和音,协和、不协和
the rate of harmony changes regular rate or irregular

三和弦 triad 基石,也会在流行音乐中出现,举例:
民谣《Oh, Susanna》;
牛仔歌曲《Streets of Laredo》,两个和音,regular rate 
平安夜,和弦
铃儿响叮当,irregular rate

和音出现在低音部,基石

bluegrass music 蓝草乐队
低音琴手,yale毕业,peter salovey,yale的教务处长 provost

琶音 appeggio
普契尼《亲爱的爸爸》也用了琶音

阿尔贝提低音 Alberti Bass,一五三五……
Mozart Piano Sonata No. 16 in C major, K. 545

重复一个八度音节,Bettoven sonata

J.S. Bach 平均律,第一首,后来Glarles Goundod加入cello,改变了此曲,降调模进
Bach: The Well-Tempered Clavier Book I, BWV 846-869, Prélude N° 1 en Do majeur..

Liszt <Lusabrious Goudole>单簧管,大调小调不断切换

Bethowvn 月光奏鸣 大三和弦,小三和弦
大 大 小 小 小
小 小 小 大 大
小 大 小 小 小
小 小 小 大

U2 Love is Blindness,irrgular
顽固音,小和弦,其中一个和弦延长了

音调 Key B flat minor
变调 modulation,很难听辨出来

Appalachian Spring

Beach Boys 假终止